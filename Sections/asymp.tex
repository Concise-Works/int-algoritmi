\section{Asymptotic Notation}

Asymptotic analysis is a method for describing the limiting behavior of functions as inputs grow infinitely.

\begin{Def}[Asymptotic]
    
    Let \(f(n)\) and \(g(n)\) be two functions. As \(n\) grows, if \(f(n)\) grows closer to \(g(n)\) never reaching, we say that \underline{``$f(n)$ is \textbf{asymptotic} to \(g(n)\).''}\\
    
    \noindent
    We call the point where \(f(n)\) starts behaving similarly to \(g(n)\) the \underline{\textbf{threshold} \(n_0\).} After this point $n_0$, \(f(n)\) follows the same general path as \(g(n)\).
\end{Def}


\begin{Def}[Big-O: (Upper Bound)]

    Let $f$ and $g$ be functions. $f(n)$ our function of interest, and $g(n)$
    our function of comparison.\\

    \noindent
    Then we say $f(n)=O(g(n))$, ``$f(n)$ \textbf{is big-O of} $g(n)$,'' if $f(n)$ 
    grows no faster than $g(n)$, up to a constant factor.
    Let $n_0$ be our asymptotic threshold. Then, for all $n\geq n_0$,
    \large
    \[0\leq f(n) \leq c\cdot g(n)\]
    \normalsize
    
    \noindent
    Represented as the ratio $\dfrac{f(n)}{g(n)}\leq c$ for all $n\geq n_0$. Analytically we write,
    \Large
    \[\lim_{n\to\infty}\dfrac{f(n)}{g(n)}< \infty\]
    \normalsize
    \noindent
    Meaning, as we chase infinity, our numerator grows slower than the denominator, bounded, never reaching infinity. 
\end{Def}

\newpage

\noindent
\textbf{Examples:}
\begin{enumerate}
    \item[(i.)] $3n^2+2n+1=O(n^2)$
    \item[(ii.)] $n^{100}=O(2^n)$
    \item[(iii.)] $\log n=O(\sqrt{n})$ 
\end{enumerate}

\begin{Proof}[$\log n = O(\sqrt{n})$]
We setup our ratio:
\[\lim_{n\to\infty}\dfrac{\log n}{\sqrt{n}}\]
\noindent
Since $\log n$ and $\sqrt{n}$ grow infinitely without bound, they are of indeterminate form $\frac{\infty}{\infty}$. We apply L'Hopital's Rule, which states
that taking derivatives of the numerator and denominator will yield an evaluateable limit:
\Large
\[\lim_{n\to\infty}\dfrac{\log n}{\sqrt{n}}=\lim_{n\to\infty}\dfrac{\frac{d}{dn}\log n}{\frac{d}{dn}\sqrt{n}}\]
\normalsize
\noindent
Yielding derivatives, $\log n = \frac{1}{n}$ and $\sqrt{n}=\frac{1}{2\sqrt{n}}$. We substitute these back into our limit:
\Large
\[\lim_{n\to\infty}\dfrac{\frac{1}{n}}{\frac{1}{2\sqrt{n}}}=\lim_{n\to\infty}\dfrac{2\sqrt{n}}{n}=\lim_{n\to\infty}\dfrac{2}{\sqrt{n}}=0\]
\normalsize
\noindent
Our limit approaches 0, as we have a constant factor in the numerator, and a growing denominator. Thus, $\log n = O(\sqrt{n})$, as $0<\infty$.
\end{Proof}

