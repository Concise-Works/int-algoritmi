\section{Interval Scheduling}
\label{sec:interval}
Scheduling problems arise in many areas, such as scheduling classes, tasks, or jobs. Interval scheduling is a type of scheduling problem where we want to maximize the number of tasks we can complete.\\

\begin{Def}[Schedule]
    
    A \textbf{schedule} is a set of tasks which we call \textbf{jobs}. Each job has a start time $s_i$ and an end time $f_i$.
    Two jobs are \textbf{compatible} if they do not overlap.
\end{Def}
\begin{figure}[h]
    \begin{center}
      \includegraphics[height=3in]{./Sections/sched/interval/interval.png}
    \end{center}
     \caption{Given jobs $a$ through $h$ we find the largest subset of mutually compatible jobs $\{b,e,h\}$.}\label{fig:interval}
\end{figure}

\newpage

\begin{Def}[Greedy Algorithm]
    
    A \textbf{greedy algorithm} is an algorithm that makes the best choice at each step. I.e.,
    it cares not about the future or big picture, only the immediate benefit, for fast computations.
\end{Def}

\begin{Note}
    \textbf{Note:} This definition becomes \textit{loose}, as we encounter problems with backtracking or multiple states. As in each state
    it makes the best choice with the information available.
\end{Note}

\textbf{Possible Approaches:} Let $s_j$ and $f_j$ be the start and finish times of job $j$.
\begin{itemize}
    \item \textbf{[Earliest Start Time]:} Consider jobs in ascending order of $s_j$.
    \item \textbf{[Earliest Finish Time]:} Consider jobs in ascending order of $f_j$.
    \item \textbf{[Shortest Interval]:} Consider jobs in ascending order of $f_j - s_j$.
    \item \textbf{[Fewest Conflicts]:} For each $j$, count the number of conflicting jobs $c_j$. \par
    \hspace{9.3em} Schedule in ascending order of $c_j$.
\end{itemize}
\noindent
We choose the \textbf{Earliest Finish Time} approach:
\begin{Proof}[Greedy Algorithm Earliest Finish Time Correctness]
    Let $i_1, i_2, \dots, i_k$ denote the set of jobs selected by the greedy algorithm.
    
    Let $j_1, j_2, \dots, j_m$ denote the set of jobs in an optimal solution, with
    \[
    i_1 = j_1, i_2 = j_2, \dots, i_r = j_r \text{ for the largest possible value of } r.
    \]
    \noindent
    We can swap $j_{r+1}$ for $i_{r+1}$ in the optimal schedule, and it will still remain compatible. We repeat swaps until $r = k$.
    It’s not possible that $m > k$ because $j_{k+1}$ is compatible with $i_k$.
    \end{Proof}
   \begin{figure}[h]
    \begin{center}
      \includegraphics[height=1.7in]{./Sections/sched/interval/interval_proof.png}
    \end{center}
     \caption{Shows that at the first divergence, $i_{r+1}$ and .}\label{fig:interval_proof}
\end{figure}

\newpage 

\noindent
\begin{theo}[Interval Scheduling \& Earliest Finish Time]
    
    Given a set of jobs $j$ with start and finish times $s_j$ and $f_j$, we can obtain an optimal like solution by scheduling in ascending order of $f_j$,
    choosing the next compatible job.
\end{theo}
\begin{figure}[h]
    \begin{center}
      \includegraphics[height=2.3in]{./Sections/sched/interval/interval_sol.png}
    \end{center}
     \caption{Solution to Figure (\ref{fig:interval}) using early finish time first, yielding $\{b,e,h\}$.}\label{fig:interval_sol}
\end{figure}

\begin{Func}[EarliestFinishTimeFirst Algorithm - \texttt{EFT($s_1, \dots, s_n, f_1, \dots, f_n$)}]
    Finds the maximum set of non-overlapping jobs based on earliest finish time.

    \vspace{.5em}
    \noindent
    \textbf{Input:} A set of jobs with start times $s_j$ and finish times $f_j$.\\
    \textbf{Output:} The maximum set of selected jobs.\\
    \begin{algorithm}[H]
        \SetAlgoLined
        sorted\_jobs $\gets$ sort($f_1, \dots, f_n$) \tcp*[f]{sort by finish time}
        $S \gets \emptyset$ \tcp*[f]{selected jobs}
        $f_{\text{last}} \gets -\infty$\;

        \For{each $j$ in sorted\_jobs}{
            \If{$f_{\text{last}} \leq s_j$}{
                $S \gets S \cup \{j\}$\;
                $f_{\text{last}} \gets f_j$\;
            }
        }
        \KwRet{$S$}
    \end{algorithm}

    \noindent
    \textbf{Time Complexity:} $O(n\log n)$ assuming our sorting algorithm is $O(n\log n)$. Then we iterate through $n$ jobs.\\
    \textbf{Space Complexity:} $O(n)$ storing the input of $n$ jobs.
\end{Func}

\newpage
\section{Interval Partitioning}
Interval partitioning generalizes our interval scheduling to multiple resources, allowing them to run in parallel.

\begin{Def}[Interval Partitioning]
    
    Given a schedule of jobs $j$ with start and finish times $s_j$ and $f_j$. We \textbf{partition} jobs into a minimal amount of $k$ resources such that no two jobs on the same resource overlap.
\end{Def}
\textbf{Scenerio: \textit{Class Scheduling}}\\
\noindent
Say we have $n$ classes and $k$ classrooms. What are the minimum number of classrooms needed to schedule all classes?\\


\noindent
\textbf{Example:} Let $n=\{a,b,c,\dots,i\}$ be classes with start and finish times. We attempt to find the minimum number of $k$ classrooms needed to schedule all classes.\\

\noindent
(1.)\label{ex:class_1}
\begin{figure}[h]
    \begin{center}
      \includegraphics[height=2in]{./Sections/sched/interval/part/class_4.png}
    \end{center}
     \caption{Though not optimal, here is a possible schedule where $k=4$.}\label{fig:class_4}
\end{figure}

\noindent
We strategies and figure out the minimum number of classrooms needed to schedule all classes in the worst-case.
We observe in Example (\ref{ex:class_1}) that $\{c,b,a\}$ strictly overlap. Moreover, there are at most $3$ classes overlapping at any time. Thus, we need at least $3$ classrooms.\\
\begin{theo}[Minimality of Interval Partitioning]
    
    Given a set of jobs $j$, $c$ conflicting tasks, and $k$ resources. We find the 
    optimal $k$ by $k = \max(c)$.
\end{theo}

\newpage
\textbf{Possible Approaches:} Let $s_j$ and $f_j$ be the start and finish times of job $j$.
\begin{itemize}
    \item \textbf{[Earliest Start Time]:} Consider jobs in ascending order of $s_j$.
    \item \textbf{[Earliest Finish Time]:} Consider jobs in ascending order of $f_j$.
    \item \textbf{[Shortest Interval]:} Consider jobs in ascending order of $f_j - s_j$.
    \item \textbf{[Fewest Conflicts]:} For each $j$, count the number of conflicting jobs $c_j$. \par
    \hspace{9.3em} Schedule in ascending order of $c_j$.
\end{itemize}
\begin{theo}[Interval Partitioning \& Earliest Start Time]

    Given a set of jobs $j$ with start and finish times $s_j$ and $f_j$, we can obtain an optimal like solution by scheduling in ascending order of $s_j$.
    If two jobs overlap, we allocate a new resource.
\end{theo}

\begin{Proof}[Classroom Allocation by Early Start Time First]
    Let $d$ be the number of classrooms that the algorithm allocates:

    \begin{enumerate}
        \item [(i.)] Classroom $d$ is opened because we needed to schedule a lecture, say $j$, that is incompatible with all $d - 1$ other classrooms.
        \item [(ii.)] These $d$ lectures each end after $s_j$.
        \item [(iii.)] Since we sorted by start time, all these incompatibilities are caused by lectures that start no later than $s_j$.
    \end{enumerate}
    \noindent
    All schedules use $\geq d$ classrooms. Thus, we have $d$ lectures overlapping at time $s_j + \epsilon$.

    \end{Proof}

    \begin{Tip}
        Though number of conflicts is the optimal number of classrooms, it tells us nothing about how to allocate them. We can use the \textbf{Earliest Start Time} as 
        it identifies our next best choice and allocates conflicts as they arise.
    \end{Tip}

    \newpage
    \begin{Func}[EarliestStartTimeFirst Algorithm - \texttt{EST($j = 1 \dots n : s_j, f_j$)}]
        Finds an optimal schedule of lectures based on their earliest start time.
        
        \vspace{.5em}
        \noindent
        \textbf{Input:} A set of lectures with start times $s_j$ and finish times $f_j$.\\
        \textbf{Output:} Assignment of lectures to rooms.\\
        \begin{algorithm}[H]
            \SetAlgoLined
            $\mathcal{A} \gets$ empty hash table \tcp*[f]{$\mathcal{A}[k]$ contains the list of lectures assigned to room $k$}
            sorted\_class $\gets$ sort($s_1, \dots, s_n$) \tcp*[f]{sort lectures by start time}
            
            \For{each $c$ in sorted\_class}{
                $k \gets$ find\_compatible\_room($c$, $\mathcal{A}$, $Q$)\;
                \If{$k$ is not None}{
                    $\mathcal{A}[k]$.add($c$)\;
                }
                \Else{
                    $d \gets$ len($\mathcal{A}$) \tcp*[f]{highest room id}
                    $\mathcal{A}[d + 1] \gets [\ ]$ \tcp*[f]{open new room}
                    $\mathcal{A}[d + 1]$.add($c$)\;
                }
            }
            \KwRet{$\mathcal{A}$}
        \end{algorithm}

        \noindent
        \textbf{Time Complexity:} $O(n\log n)$ assuming our sorting algorithm is $O(n\log n)$. Then we iterate through $n$ jobs.\\
        \textbf{Space Complexity:} $O(n)$ storing the input of $n$ jobs.
    \end{Func}

    
    
    
